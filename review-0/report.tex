\documentclass[12pt,a4paper]{article}
\usepackage{ssnreview}
\usepackage{float}
\usepackage{url}
\usepackage{alltt}
\usepackage{longtable}
%\usepackage{mathtools}
\usepackage{algorithm2e}[1]

\begin{document}

\ptitle{Sample-Efficiency in Complex Environments using Reinforcement Learning}
\setnauthors{3}
\setauthorone{Nimish S}{312217104098}
\setauthortwo{Sadhana Smruthi S}{312217104134}
\setauthorthree{Avi K Natesan}{312217104022}
\semester{7}
\guide{Dr. R. Priyadharsini}
\review{0}
\reviewdate{30 October 2020}
\reviewtitle
\hrule

\section{Abstract}
Current Reinforcement Learning algorithms excel at performing in
simple environments. However, when it comes to complex environments 
that are analogous to real world environments, they are impractical to use 
as sample-inefficiency becomes more prominent due to the curse of
dimensionality. This makes reinforcement learning impractical to 
use in many real-world situations as many  of these environments are 
complex, hierarchical \& sparse in  nature. Superior exploration techniques 
like Curiosity-driven Exploration have proven to improve sample efficiency 
in simple environments. We propose to use such techniques in combination 
with existing RL algorithms to successfully navigate a complex 
environment.

\section{Introduction}
Reinforcement learning algorithms aim at learning policies for achieving 
target tasks by maximizing rewards provided by the environment. Reinforcement
learning combined with neural networks has recently led to a wide range of 
successes in learning policies for sequential decision-making problems. From a
practical standpoint, RL is currently used in simple environments where PID 
\textit{(Proportional Integral Derivative)} controllers cannot be easily applied
(e.g. simple robotic control, HVACs \& autonomous vehicles). Currently, 
it is crucial for the environments to be designed in such a way as to encourage 
learning.

However, in most real-world applications, there are three important environmental
characteristics most reinforcement learning algorithms struggle with - sparseness of the reward, 
hierarchy in tasks and complexity of the environment. The 
rewards supplied to the agent are extremely sparse or sometimes missing 
altogether. The actions  to be taken are hierarchical in nature (i.e. 
each goal state does not result in termination, rather opens the possibility 
to new goal states). The state space is extremely high dimensional leading 
to a lot of time spent on extracting useful features from higher-level 
representations. A combination of all these problems results in extremely 
sample-inefficient performance of existing algorithms.

In 2018, the team at OpenAI tried to attain human-level control in an extremely 
complex environment called Dota 2. It is a strategic real-time video game with 
170,000 possible actions in each time-step that took humans approximately 600 
hours to learn. The rewards too were very sparse with the agent attaining 
significant reward once every 5-10 minutes. A \textit{Proximal Policy Optimization} 
agent was trained over a period of 10 months. It cost 7.5 million dollars to 
train and was trained using 128,000 CPU cores and 256 GPUs. It read in 1,048,576 
observations every second which is an unimaginably large number of samples. This 
entire setup is simply impractical for use on a large scale.

Sample-inefficiency is just the result of the explore-exploit dilemma 
in practice. In order to attain the maximum possible reward, sufficient
exploration must be done to identify the best possible sequence of decisions
in a given situation and then the agent must exploit its knowledge of the environment. There's a fine line between both as more exploration
would take a lot of time to experience states further down the hierarchy and pure 
exploitation will result in the agent getting stuck in a local maxima. 

Currently for complex environments, algorithms explore using a naive method 
based on adding random noise to the action probabilities of the agent. We propose 
to use intelligent exploration techniques introduced in much simpler environments in combination with powerful RL algorithms to significantly improve sample efficiency in complex environments. In order to test this theory, we plan to use a complex environment called MineRL which is based on a 3D sandbox survival video game
called Minecraft. The game focuses on the player collecting resources and crafting 
materials in order to survive. We plan to test this theory first using the Super 
Mario Bros environment which is a 2D environment but presents its own set of 
complications due to its large non-repetitive state space and complex action space.

\section{Literature survey}
Mnih, et al. \cite{dqn} introduced the off-policy DQN \textit{(Deep Q-Network)} 
agent which used neural networks to approximate the values of continuous state 
spaces. Till then, the applicability of RL has been limited to domains with fully 
observed, low-dimensional state spaces with discrete actions. The DQN agent could 
tackle significantly more complex environments than possible before this point. The 
agent was tested on the Atari 2600 platform which consisted of 49 games. The 
observation was given in the form of pixel representations of the game at every 
time step. The performance of the agent was scaled between 100\% (professional 
human player) and 0\% (random agent). The experiments showed that with a single 
set of hyperparameters, the agent was able to achieve superhuman ( $>$ 200\%) 
performance in 14 games, professional level ( $>$ 100\%) performance in 9 games
and above average ( $>$ 75\%) performance in 6 games. This proved the robustness 
of the algorithm to different environments and also proved that the agents could 
learn meaningful state representations from high-level features like pixels. 
DQfD \textit{(Deep Q-learning from Demonstrations)} \cite{dqdf}, leverages small 
sets of demonstration data to massively accelerate the learning process. It is 
able to automatically assess the necessary ratio of demonstration data while 
learning thanks to the Prioritized Experienced Replay \cite{per} mechanism. 

Lillicrap, et al. \cite{ddpg} presented a model-free, off-policy actor-critic
algorithm called DDPG \textit{(Deep Deterministic Policy Gradient)} that can 
operate over continuous action spaces unlike DQN. DDPG can learn competitive 
policies using low-dimensional observations (e.g. Cartesian coordinates or joint 
angles) and even directly from pixels using the same hyperparameters and network 
structure. Robotic control environments, where the agent could control all the 
joints of a robotic arm in 7 degrees of freedom and gait-simulation environments, 
where the agent could control the joints of a bipedal and a quadripedal entity 
were used to evaluate the performance of the algorithm. In both cases, the 
algorithm was able to converge close to an optimal policy within 2.5 million 
steps. Since DDPG can only work in continuous action spaces and DQN can only 
work with discrete actions, there is no baseline for comparison between the two. 
However, it has been generally proven via experiments that in most cases DDPG 
could converge to a better policy whereas DQN was more sample efficient. DDPG 
tended to overestimate values of certain states leading it to converge to 
suboptimal policies. Fujimoto, et al. \cite{td3} introduced
a modified version of the DDPG algorithm called the TD3 \textit{(Twin Delayed Deep Deterministic Policy Gradient)} which addressed the overestimation bias of the DDPG, allowing it to converge faster to better policies.
\linebreak

Pathak et al. \cite{cdl} introduced the ICM \textit{(Intrinsic Curiosity Module)} 
architecture as an alternative to random exploration techniques in environments with 
sparse or no extrinsic rewards. RL algorithms were predominantly trained in environments that provided a continuous supply of extrinsic rewards. ICM architecture enabled training in environments that resembled real-world scenarios due to their lack of extrinsic rewards. Curiosity also aided agents in exploring environments for new knowledge and learning skills that might help in future tasks. ICM architecture 
was tested in two environments for 3 different settings - sparse extrinsic rewards, 
no extrinsic rewards and novel scenarios. The sparse extrinsic reward experiment was 
performed on VizDoom by varying the degree of reward sparsity. The curious agent learnt much faster in the 'dense' and sparse reward case. The baseline agent failed to solve the task in the sparse reward case while both failed to solve the 'very sparse' reward case. The no reward setting saw agents successfully explore and navigate environments with no extrinsic rewards in Super Mario Bros and VizDoom environments. The agent also discovered winning behaviours. The novel-scenario setting evaluated 
generalization of agent behaviours in two ways: `as is` (no further learning) and 
fine-tuning the policies. Good generalization was observed in 'as-is' for scenarios 
that weren't significantly different visually. Fine-tuning with curiosity helped 
tackle the mentioned drawbacks.

Andrychowicz et al. \cite{her} introduced a novel technique called HER 
\textit{(Hindsight Experience Replay)} that enabled sample-efficient learning by 
combining it with off-policy RL algorithms to problems with sparse and binary 
rewards. A standard RL algorithm would learn little to nothing from a sequence of 
actions that lead to an unsuccessful state while HER encourages a replay of the same
actions for a different goal giving rise to faster learning. Transition tuples 
encountered during training are stored in a replay buffer for harvesting 
information using different goals. HER was tested in combination with off-policy 
algorithms like DDPG for 3 different robotic control tasks - pushing, sliding and pick-and-place.
For HER, each transition was stored in the replay buffer twice: for the goal used to 
generate the episode and the goal corresponding to the final state. DDPG without 
HER was unable to solve any of the tasks while DDPG with HER solved them all, almost
perfectly. DDPG with HER improved 
performance even when the goal state was identical in all episodes. The agent 
learned faster when training episodes had multiple goals. When reward functions
were shaped (and not the previous binary), both DDPG and the HER supported agents
were unable to solve any of the tasks. An analysis of alternative replay goals
tested future, episode and random goal strategies. It showed
that the most valuable goals for replay were ones that were going to be achieved
in the near future. HER was also tested on a physical fetch robot without fine-tuning
and performed well on the pick-and-place task.

\section{Proposed System}

\begin{figure}[H]
  \centering
  \includegraphics[scale=0.55]{review-0-prev-architecture.png}
  \caption{Typical RL Systems}
  \label{fig:typical-system}
\end{figure}

\begin{figure}[H]
  \centering
  \includegraphics[scale=0.55]{review-0-architecture.png}
  \caption{Proposed System Architecture}
  \label{fig:proposed-system}
\end{figure}

Typical modern  off-policy RL systems shown in Figure~\ref{fig:typical-system} learn from 
experience tuples sampled randomly from a standard Replay Buffer.

The proposed system in Figure~\ref{fig:proposed-system} is a Curiosity-Driven Off-Policy agent which includes 
the ICM \textit{(Intrinsic Curiosity Module)} coupled with a 
DQN based off-policy learner and a HER \textit{(Hindsight Experience Replay)} Buffer. 
The system is aimed at creating a sample-efficient method to train an agent to
navigate complex environments.

The agent initially observes the state $s$ of the environment and performs an action $a$ 
for which the environment returns a new state $s'$ and an extrinsic reward $r^e$. This experiences
tuple consisting of a state, action, reward and next state  $(s, a, r^e, s')$ is stored in a 
replay buffer which is a store for experience tuples. Independently, the buffer is sampled 
by the ICM and it generates an intrinsic reward signal $r^i$ which is combined with the extrinsic reward and given to the agent to perform the TD \textit{(temporal difference)} update.
Based on this newly constructed reward signal, the TD update is performed. With these newly calculated
estimates, the agent will start to explore previously unseen states and find better 
solutions to the environment.

The environment to conduct the experiment in is 2D Super Mario Bros environment which has a continuous, non-repetitive state space in the form of pixel data with an 
extremely sparse reward system. The objective in this environment is to navigate to the 
end of the level while avoiding all obstacles. The complexity of this environment is attributed
to the fact that in a single level, a state will never occur more than once due to the
uni-directional nature of the level. This environment will be used as a baseline for comparison
with a much more complex 3D environment called MineRL with hierarchical tasks where the 
goal is to craft specific resources by gathering raw materials from the environment. The choice of the agent, 
DDDQN \textit{(Dueling Double Deep Q-Network)} \cite{duel}\cite{double} or DQfD \textit{(Deep Q-Learning from Demonstrations)} 
depends on the environment.

\section{Module Split-Up}
Our proposed system will include three modules: an Off-Policy agent, an Intrinsic Curiosity Module(ICM) and a Replay Buffer.
An Off-Policy agent is preferred as it is known to be sample efficient in nature. We plan on using a
variation of the DQN agent depending on the environment. The ICM will be used to efficiently 
explore the sparse-reward settings. It will enable generalized policies. The ICM will be used in
the pipeline between the Replay Buffer and the agent to generate a modified intrinsic reward signal.
The third module we will be employing is the Hindsight Experience Replay (HER) Buffer. As
opposed to randomly sampling a Replay Buffer, HER makes efficient use of the samples stored in the buffer. With
the aid of the HER and the ICM, we plan on improving the sample efficiency of the Off-Policy agent, even in complex
environments.



% \subsection{Off-Policy Agent}
% An Off-Policy agent was chosen for the project for two reasons. The relatively sample-efficient
% nature of such algorithms due to the use of a Replay Buffer and due to the availability of 
% human-demonstrations for the second environment which requires the aforementioned Replay Buffer.
% The algorithm to be used will be a slight modification of the DQN called the DDDQN which has a 
% better performance overall. The second environment will use a DQfD which doesn't have to interact with environment
% as much as the batches of data are already present. The Q-Network in the agents will take in input in the 
% form of stacked grayscale pixel-data (e.g. frames from 4 consecutive time steps stacked to form a 4 channel image) 
% to preserve the characteristics of the change in the environment over time. The network produces two outputs. A single
% value depicting the overall value of the state $V(s)$ and a tensor of action advantage values for every action possible 
% in the state $A^\pi(s,a) \forall a \in A$. The loss function used will be Huber Loss (SmoothL1Loss) as it is less sensitive to 
% outliers in data than the squared error loss. This output is indirectly used by an $\epsilon$-greedy policy to explore or 
% exploit its current knowledge of the environment where the $\epsilon$ value
% is reduced from 1.00 (completely random action) to 0.01 (99\% deterministic action) over 1,000,000
% interactions with the environment.

% \subsection{Simplified Intrinsic Curiosity Module}
% This module is to be used in the pipeline in between the Replay Buffer and the agent to generate
% a modified intrinsic reward signal as opposed to directly sending the batch of data from the Replay Buffer
% to the agent for the TD update. The rewards stored in the Replay Buffer are from the environment and are sparse
% and so the ICM generates a modified reward using which the TD update is performed by the agent.
% This module is also used to make up for the drawback of DQN based agents which is their inability to generalize
% to unseen states as evidence suggests that this module improves the ability of an agent to generalize.
% The ICM consists of a neural network that takes in a state $s_t$ and an action $a_t$ and predicts the next state $s'_{t+1}$
% which is compared with the actual next state $s_{t+1}$. The loss function used for this network is a mean squared error loss. 
% The magnitude of the loss here shows its level of knowledge about the dynamics of the 
% environment. Higher the loss, higher the intrinsic reward which incentivizes a sense of 
% "curiosity" in the agent to revisit the visited state and explore further.

% \subsection{Replay Buffer}
% The Replay Buffer is a collection of states $s$, actions $a$, extrinsic rewards $r^e$ and 
% next states $s'$ which constitutes an experience tuple. The Buffer is uniformly sampled 
% to create a batch of uncorrelated experience tuples in a random order. There are multiple 
% advantages
% to doing this. First, each step of experience is potentially used in many weight 
% updates, which allows for greater data efficiency. Second, learning directly from
% consecutive samples is inefficient, owing to the strong correlations between the 
% samples; randomizing the samples breaks these correlations and therefore reduces 
% the variance of the updates. Third, when learning on-policy the current parameters 
% determine the next data sample that the parameters are trained on. For example, if the
% maximizing action is to move left then the training samples will be dominated by
% samples from the left-hand side; if the maximizing action then switches to the 
% right then the training distribution will also switch. 
% It is easy to see how unwanted feedback loops may arise and the parameters could
% get stuck in a poor local minimum, or even diverge catastrophically. The use of a Replay
% Buffer avoids such scenarios.

\begin{thebibliography}{99}
\bibliographystyle{plain}

  \bibitem[1]{dqn}Mnih, V., Kavukcuoglu, K., Silver, D. et al. 
  {\em Human-level control through deep reinforcement learning.} 
   Nature 518, 529–533 (2015).

  \bibitem[2]{dqdf}Hester, T., Vecerik, M., Pietquin, 0., et al. 
  {\em Deep Q-learning from Demonstrations.}
  ArXiv (2017).

  \bibitem[3]{per}Schaul, T., Quan, J., Antonoglou, I., and Silver, D.
  {\em Prioritized Experience Replay.}
  ArXiv (2015).

  \bibitem[4]{ddpg}Lillicrap, T., Hunt, J., Pritzel, A. et al.
  {\em Continuous control with deep reinforcement learning.}
  ArXiv (2015).

  \bibitem[5]{td3}Fujimoto, S., Hoof, H., Meger, D.
  {\em Addressing Function Approximation Error in Actor-Critic Methods.}
  ArXiv (2018).

  \bibitem[6]{cdl} Pathak, D., Agrawal, P., Efros, A., Darrell, T.
  {\em Curiosity-Driven Exploration by Self-Supervised Prediction.}
  IEEE Conference on Computer Vision and Pattern Recognition Workshops (CVPRW) (2017).

  \bibitem[7]{her} Andrychowicz, M., Wolski, F., Ray, A.,
  {\em Hindsight Experience Replay.}
  Advances in Neural Information Processing Systems 30 (NIPS) (2017).

  % \bibitem[8]{cdl2} B. Li, T. Lu, J. Li, N. Lu, Y. Cai and S. Wang, 
  % {\em Curiosity-Driven Exploration for Off-Policy Reinforcement Learning Methods.}
  % IEEE International Conference on Robotics and Biomimetics (ROBIO) (2019)

  \bibitem[8]{duel} Hasselt, H., Guez, A. and Silver, D., 
  {\em Dueling Network Architectures for Deep Reinforcement Learning.}
  ArXiv (2015).

  \bibitem[9]{double} Wang, Z., Freitas, N. and Lanctot, M. 
  {\em Deep Reinforcement Learning with Double Q-learning.}
  ArXiv (2015).

\end{thebibliography}
	
\end{document}
