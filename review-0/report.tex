\documentclass[12pt,a4paper]{article}
\usepackage{ssnreview}
\usepackage{float}
\usepackage{url}
\usepackage{alltt}
\usepackage{longtable}
%\usepackage{mathtools}
\usepackage{algorithm2e}[1]
\usepackage{tikz}
\usetikzlibrary{shapes,arrows}
\usetikzlibrary{trees}

\begin{document}

\ptitle{Sample-Efficiency in Complex Environments using Reinforcement Learning}
\setnauthors{3}
\setauthorone{Nimish S}{312217104098}
\setauthortwo{*****}{312217104***}
\setauthorthree{*****}{312217104***}
\semester{7}
\guide{*****}
\review{0}
\reviewdate{24 October 2020}
\reviewtitle
\hrule

\section{Abstract}
Current Reinforcement Learning algorithms excel at performing in
non-hierarchial environments. However, when it comes to complex,
hierarchial \& sparse environments  they are impractical to use 
as sample-inefficiency becomes more prominent due to the curse of
dimensionality. This makes reinforcement learning impractical to 
use in many real-world situations as many  of these environments are 
complex, hierarchial \& sparse in  nature. Superior exploration techniques 
like Curiosity-driven Exploration have proven to improve sample efficiency 
in simple environments. We propose to use such techniques in combination 
with RL algorithms like DDPG and DQN to succesfully navigate a complex 
environment.

\section{Introduction}
Reinforcement learning algorithms aim at learning policies for achieving 
target tasks by maximizing rewards provided by the environment. Reinforcement
learning combined with neural networks has recently led to a wide range of 
successes in learning policies for sequential decision-making problems. From a
practical standpoint, RL is currently used in simple environments where PID 
\textit{(Proportional Integral Derivative)} controllers cannot be easily applied
(e.g. simple robotic control, HVACs \& autonomous vehicles). Currently, 
it is crucial for the environments to be designed in such a way as to encourage 
learning.

However, in most real-world applications, there are three important environmental
characteristics most reinforcement learning algorithms struggle with. The 
rewards supplied to the agent are extremely sparse or sometimes missing 
altogether. The actions  to be taken are hierarchial in nature (i.e. 
each goal state does not result in termination, rather opens the possibility 
to new goal states). The state space is extremely high dimensional leading 
to a lot of time spent on extracting useful features from higher-level 
representations. A combination of all these problems results in extremely 
sample-inefficient performance of existing algorithms.

In 2018, the team at OpenAI tried to attain human-level control
in an extremely complex environment called Dota2. It is a 
strategic real-time video game with 170,000 possible actions in each
time-step that took humans approximately 600 hours to learn. The rewards 
too were very sparse with the agent attaining significant reward once 
every 5-10 minutes. A \textit{Proximal Policy Optimization}
agent was trained over a period of 10 months. It cost 7.5 million 
dollars to train and was trained using
128,000 CPU cores and 256 GPUs. It read in 1,048,576 observations 
every second which is an unimaginably large number of samples. 
This entire setup is simply impractical for use on a large scale.

Sample-inefficiency is just the result of the explore-exploit dilemma 
in practice. In order to attain the maximum possible reward, sufficient
exploration must be done to identify the best possible sequence of decisions
in a given situation and then the agent must exploit its knowledge of the environment.
There is a fine line between exploration and exploitation as more exploration
would take a lot of time to experience states further down the hierarchy and pure 
exploitation will result in the agent getting stuck in a local maxima. 

Currently for complex enironments, algorithms 
explore using a naive method based on adding random noise to the action 
probabilities of the agent. We propose to use intelligent exploration 
techniques introduced in much simpler environments in combination with 
powerful RL algorithms to significantly improve sample efficiency in 
complex environments. In order to test this theory, we plan to use a complex
environment called MineRL which is based on a 3D sandbox survival video game
called Minecraft. The game focuses on the player collecting resources and crafting materials
in order to survive. We plan to test this theory first using the SuperMarioBros 
environment which is a 2D environment but presents its own set of complications 
due to its large non-repetitive state space and complex action space.

\section{Literature survey}
Minh, et al. \cite{dqn} introduced the off-policy DQN \textit{(Deep Q-Network)} agent which
used neural networks to aproximate the values of continuous state spaces. Till then, the applicability of RL
has been limited to domains with fully observed, low-dimensional state spaces with
discrete actions. The DQN agent could tackle significantly more complex environments than possible before
this point. The agent was tested on the Atari 2600 platform which consisted of
49 games. The observation was given in the form of pixel representations of the game at every time step.
The performance of the agent was scaled between 100\% (professional human player) and 0\% (random agent).
The experiments showed that with a single set of hyperparameters, the agent
was able to achieve superhuman ( $>$ 200\%) performance in 14 games, professional level
( $>$ 100\%) performance in 9 games and above average ( $>$ 75\%) performance in 6 games.
This proved the robustness of the algorithm to different environments and also proved that
the agents could learn meaningful state representations from high-level features like pixels.
DQfD \textit{(Deep Q-learning from Demonstrations)} \cite{dqdf}, leverages small sets of demonstration data to 
massively accelerate the learning process. It is able to automatically assess the necessary ratio of demonstration data while 
learning thanks to the Prioritized Experienced Replay \cite{per} mechanism.

Lillicrap, et al. \cite{ddpg} presented a model-free, off-policy actor-critic
algorithm called DDPG \textit{(Deep Deterministic Policy Gradient)} that can operate over continuous action spaces unlike DQN.  
DDPG can learn competitive policies using low-dimensional 
observations (e.g. cartesian coordinates or joint angles) and even directly from pixels 
using the same hyperparameters and network structure. Robotic control environments, where 
the agent could control all the joints of a robotic arm in 7 degrees of freedom and gait-simulation environments, 
where the agent could control the joints of a bipedal and a quadripedal entitiy were used to evaluate the performance
of the algorithm. In both cases, the algorithm was able to converge close to an optimal policy within 2.5 million 
steps. Since DDPG can only work in continuous action spaces and DQN can only work with discrete actions, there
is no baseline for comparison between the two. However, it has been generally proven via experiments that
in most cases DDPG could converge to a better policy whereas DQN was more sample efficient. DDPG tended to overestimate
values of certain states leading it to converge to suboptimal policies. Fujimoto, et al. \cite{td3} introduced
a modified version of the DDPG algorithm called the TD3 \textit{(Twin Delayed Deep Deterministic Policy Gradient)} which addressed the 
overestimation bias of the DDPG, allowing it to converge faster to better policies.



\section{Environments}


\begin{thebibliography}{99}
\bibliographystyle{plain}

\bibitem[1]{dqn}Mnih, V., Kavukcuoglu, K., Silver, D. et al. 
  {\em Human-level control through deep reinforcement learning.} 
   Nature 518, 529–533 (2015).

\bibitem[2]{dqdf}Hester, T., Vecerik, M., Pietquin, 0., et al. 
  {\em Deep Q-learning from Demonstrations.}
  ArXiv (2017).

\bibitem[3]{per}Schaul, T., Quan, J., Antonoglou, I., and Silver, D.
  {\em Prioritized Experience Replay.}
  ArXiv (2015).

\bibitem[4]{ddpg}Lillicrap, T., Hunt, J., Pritzel, A. et al.
  {\em Continuous control with deep reinforcement learning.}
  ArXiv (2015).

\bibitem[5]{td3}Fujimoto, S., Hoof, H., Meger, D.
  {\em Addressing Function Approximation Error in Actor-Critic Methods.}
  ArXiv (2018).

\end{thebibliography}


	
\end{document}
